To learn the required knowledge artificial intelligence needs a source of data to work with.
These data will almost exclusively consist of scientific papers and elementary school books. 
Currently two separate directions are being pursued, image recognition and lexical analysis.

Both streams will be explained in this report, but the focus will be put on the latter.
The first will try to retrieve images from scientific papers and correctly label it by using the caption that is included with the table or figure.
In the second method the main idea is to form a lexicon to which text will be mapped and converted into logical rules that can be interpreted by the AI.
It can use these data to form Knowledge Graphs.
While answering questions the keywords of the text will be extracted and filled into the Knowledge graph.
The multiple choice questions will be reformed to multiple true or false questions. 
Every answer will lead to a percentage which will define how true the statement was.
The answer with the largest percentage is considered to be the best answer for this question.
