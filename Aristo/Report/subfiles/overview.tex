To learn the required knowledge, the artificial intelligence needs a source of data to work with.
These data will almost exclusively consist of scientific papers and elementary school books. 
Currently two separate directions are being pursued for the Aristo system: image recognition \cite{pdffigures2} and text based knowledge extraction \cite{probseman,sciencequestions}.

Both streams will be explained in this report, but the focus will be put on the latter.
The first will try to retrieve images from scientific papers and correctly label it by using the caption that is included with the table or figure.
In the second method the goal is to extract logical statements from the text. When taking the test, the questions are translated into logical queries.
The questions all use a multiple-choice format. This means that the system can focus on determining the most likely answer from the given possibilities.
By matching every possibility to the concepts and relations extracted from the text, the support for each answer can be quantified by measuring how well it fits within the knowledge that's available.
The answer with the best support is considered to be the best answer for this question.
