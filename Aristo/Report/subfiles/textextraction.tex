Texts from books and other sources often contain a large amount of information. Natural languages are flexible enough to communicate complex concepts, intricate relations and more. However, AI systems currently have a hard time accessing and reasoning with the knowledge contained in texts.
\\
Even though interpreting natural language is, so far, an unsolved problem. The limited scope of solving science tests brings a solution within reach. A significant part of the knowledge can be expressed with relations like ``X causes Y'', ``X is part of Y'', ``X is an example of Y'' or ``X [verb] Y''. The algorithm by Clark et al. \cite{construction} uses a hand-crafted set of extraction rules to generate a set of expressions from the text, where relations of interest are expressed. Figure \ref{fig:extraction} shows an example of this. Extractions are a semi-formal data structure, but they can easily be translated into formal representation. Many of these rules combined form a knowledge base.

\begin{figure}
\noindent\fbox{%
    \parbox{\columnwidth}{%
       ''Mechanical energy is produced when two objects move together.''
       \center{$\Downarrow$}
       \\
(``two objects''/?x ``produce'' ``Mechanical energy'') ``when'' /CONDITION \\(``two objects'' /?x ``move'' ``'' [ ``together'' ])
    }%
}
\caption{An example of an extraction}
\label{fig:extraction}
\end{figure}