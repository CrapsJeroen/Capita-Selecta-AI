February 1996, \textbf{Deep Blue}\footnote{The Chess playing computer designed by IBM.} wins the first ever game of Chess as a computer against the world champion at the time Garry Kasparov.
March 2016, \textbf{AlphaGo} defeats 18-time world champion Lee Sedol in a five-game Go match with a score of 4-1.

Originally the biggest test for an artificial intelligence was the Turing test.
The test requires that a human being is unable to distinguish the machine from another human being during a conversation where it has to answer to questions.
A limitation of this test is that there is no objective way to measure the progress towards the goals of AI \cite{honorstudent}.

Computers have been able to perform tasks better than humans like path planning, finding patterns and games. After 20 years of progress computers are now able to defeat the human mind at very complex games, but can it solve the same questions as asked in a 4th grade exam?

To conclude the answer is no, but current scientific research is trying to change this as this is being seen as a key component of any new measurement of artificial intelligence \cite{honorstudent}.

There are several reasons behind this.
It has all the requirements of a test \cite{honorstudent}: Accessible, Comprehendible, Measurable and offer a graduated progression for simple everyday things to deeper understanding of subjects.
Also to answer these questions a significant improvement in language understanding and the modelling of the world are be required. In this report we will be discussing some of the current methods that are being used to improve the performance of AI on these kind of tests.