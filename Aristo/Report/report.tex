%%%%%%%%%%%%%%%%%%%%%%%%%%%%%%%%%%%%%%%%%
% Journal Article
% LaTeX Template
% Version 1.4 (15/5/16)
%
% This template has been downloaded from:
% http://www.LaTeXTemplates.com
%
% Original author:
% Frits Wenneker (http://www.howtotex.com) with extensive modifications by
% Vel (vel@LaTeXTemplates.com)
%
% License:
% CC BY-NC-SA 3.0 (http://creativecommons.org/licenses/by-nc-sa/3.0/)
%
%%%%%%%%%%%%%%%%%%%%%%%%%%%%%%%%%%%%%%%%%

%----------------------------------------------------------------------------------------
%	PACKAGES AND OTHER DOCUMENT CONFIGURATIONS
%----------------------------------------------------------------------------------------

\documentclass[twoside,twocolumn]{article}

\usepackage{blindtext} % Package to generate dummy text throughout this template 

\usepackage[sc]{mathpazo} % Use the Palatino font
\usepackage[T1]{fontenc} % Use 8-bit encoding that has 256 glyphs
\linespread{1.05} % Line spacing - Palatino needs more space between lines
\usepackage{microtype} % Slightly tweak font spacing for aesthetics

\usepackage[english]{babel} % Language hyphenation and typographical rules

\usepackage[hmarginratio=1:1,top=32mm,columnsep=20pt]{geometry} % Document margins
\usepackage[hang, small,labelfont=bf,up,textfont=it,up]{caption} % Custom captions under/above floats in tables or figures
\usepackage{booktabs} % Horizontal rules in tables

\usepackage{enumitem} % Customized lists
\setlist[itemize]{noitemsep} % Make itemize lists more compact

\usepackage{abstract} % Allows abstract customization
\renewcommand{\abstractnamefont}{\normalfont\bfseries} % Set the "Abstract" text to bold
\renewcommand{\abstracttextfont}{\normalfont\small\itshape} % Set the abstract itself to small italic text

\usepackage{titlesec} % Allows customization of titles
\renewcommand\thesection{\Roman{section}} % Roman numerals for the sections
\renewcommand\thesubsection{\roman{subsection}} % roman numerals for subsections
\titleformat{\section}[block]{\large\scshape\centering}{\thesection.}{1em}{} % Change the look of the section titles
\titleformat{\subsection}[block]{\large}{\thesubsection.}{1em}{} % Change the look of the section titles



\usepackage{titling} % Customizing the title section
\usepackage{subfiles} % For subfiles in the main file
\usepackage{hyperref} % For hyperlinks in the PDF
\usepackage{graphicx}
\graphicspath{{./images/}}
\usepackage{todonotes} %Used for the figure placeholders
\usepackage{ifthen}
\usepackage{parskip}
\usepackage{caption}
\usepackage{listings}
\usepackage{tabu}
\usepackage{rotating}

%----------------------------------------------------------------------------------------
%	TITLE SECTION
%----------------------------------------------------------------------------------------

\setlength{\droptitle}{-4\baselineskip} % Move the title up

\pretitle{\begin{center}\Huge\bfseries} % Article title formatting
\posttitle{\end{center}} % Article title closing formatting
\title{Community Detection: Metaheuristic} % Article title
\author{%
\textsc{Jeroen Craps} \\[1ex] % Your name
\normalsize KU Leuven \\ % Your institution
\normalsize \href{mailto:jeroen.craps@student.kuleuven.be}{jeroen.craps@student.kuleuven.be} % Your email address
\and % Uncomment if 2 authors are required, duplicate these 4 lines if more
\textsc{Jorik De Waen} \\[1ex] % Second author's name
\normalsize KU Leuven \\ % Second author's institution
\normalsize \href{mailto:jorik.dewaen@student.kuleuven.be}{jorik.dewaen@student.kuleuven.be} % Second author's email address
}
\date{\today} % Leave empty to omit a date
%\renewcommand{\maketitlehookd}{%
%\begin{abstract}
%\noindent \blindtext % Dummy abstract text - replace \blindtext with your abstract text
%\end{abstract}
%}

%----------------------------------------------------------------------------------------

\begin{document}

% Print the title
\begin{titlepage}
    \newpage
    \thispagestyle{empty}
    \frenchspacing
    \hspace{-0.2cm}
    \includegraphics[height=3.4cm]{sedes}
    \hspace{0.2cm}
    \rule{0.5pt}{3.4cm}
    \hspace{0.2cm}
    \begin{minipage}[b]{8cm}
        \Large{KULeuven}\smallskip\newline
        \large{}\smallskip\newline
        \textbf{Department of\newline Computer Science}\smallskip
    \end{minipage}
    \hspace{\stretch{1}}
    \vspace*{3.2cm}\vfill
    \begin{center}
        \begin{minipage}[t]{\textwidth}
            \begin{center}
                \LARGE{\rm{\textbf{\uppercase{Advanced Programming Languages \\ for A.I. (H02A8a)}}}}\\
                \Large{\rm{Project: Sudoku \& Shikaku}}
            \end{center}
        \end{minipage}
    \end{center}
    \vfill
    \hfill\makebox[8.5cm][l]{%
        \vbox to 7cm{\vfill\noindent
            \ifthenelse{\boolean{anonymize}}{%
                {\rm \textbf{Anonymized}}\\
                {\rm Academic year 2015--2016}
            }{%
                {\rm \textbf{Jeroen Craps (r0292642)}}\\
                {\rm \textbf{Jorik De Waen (r0303087)}}\\ [2mm]
                {\rm Academic year 2015--2016}
            }
        }
    }
\end{titlepage}


%----------------------------------------------------------------------------------------
%	ARTICLE CONTENTS
%----------------------------------------------------------------------------------------

\section{Introduction}

February 1996, \textbf{Deep Blue}\footnote{The Chess playing computer designed by IBM.} wins the first ever game of Chess as a computer against the world champion at the time Garry Kasparov.
March 2016, \textbf{AlphaGo} defeats 18-time world champion Lee Sedol in a five-game Go match with a score of 4-1.

Originally the biggest test for an artificial intelligence was the Turing test.
The test requires that a human being is unable to distinguish the machine from another human being during a conversation where it has to answer to questions.
A limitation of this test is that there is no objective way to measure the progress towards the goals of AI \cite{honorstudent}.

Computers have been able to perform tasks better than humans like path planning, finding patterns and playing games. After 20 years of progress, computers are now able to defeat the human mind at very complex games, but can it solve the same questions as asked in a 4th grade exam?

Even though 4th grade exams are trivial to solve for humans, they present an enormous challenge for our current AI systems. Researchers are actively working on this problem as this is seen as a key component of any new measurement of artificial intelligence \cite{honorstudent}.

There are several reasons behind this.
It has all the requirements of a test \cite{honorstudent}: Accessible, Comprehendible, Measurable and offer a graduated progression for simple everyday things to deeper understanding of subjects.
Also to answer these questions a significant improvement in language understanding and the modelling of the world are be required. In this report we will be discussing some of the current methods that are being used to improve the performance of AI on these kind of tests.

%------------------------------------------------

\section{Overview}

To learn the required knowledge, the artificial intelligence needs a source of data to work with.
These data will almost exclusively consist of scientific papers and elementary school books. 
Currently two separate directions are being pursued for the Aristo system: image recognition \cite{pdffigures2} and text based knowledge extraction \cite{probseman,sciencequestions}.

Both streams will be explained in this report, but the focus will be put on the latter.
The first will try to retrieve images from scientific papers and correctly label it by using the caption that is included with the table or figure.
In the second method the goal is to extract logical statements from the text. When taking the test, the questions are translated into logical queries.
The questions all use a multiple-choice format. This means that the system can focus on determining the most likely answer from the given possibilities.
By matching every possibility to the concepts and relations extracted from the text, the support for each answer can be quantified by measuring how well it fits within the knowledge that's available.
The answer with the best support is considered to be the best answer for this question.


%------------------------------------------------

\section{Image retrieval}

%------------------------------------------------

\section{...}

%------------------------------------------------

\section{Conclusion}

\section{Conclusion}
\label{sec:conclusion}
Our experiments show that the effectiveness of reducing the graph is highly dependent on how interconnected the graph is. The advantage of reducing the graph is not only in execution time, as we expected, but it also significantly improves the quality of the initial random population for the genetic algorithm. This advantage is clear in graphs with on the order of hundreds of vertexes or more, and proportional with the amount reduction that was done. There are also obvious performance benefits for medium-sized graphs of several thousands of vertexes. However, for larger graphs with tens of thousands of vertexes, the genetic algorithm becomes ineffective when combined with the short execution times we allowed. The reduced graph still has a significant advantage, but only due to better initialization.
\par
Because of this we conclude that our graph reduction preprocessing can be very useful in certain scenarios. When graphs are highly interconnected, reducing the graph improves both the execution time and the score of the result. While the reduced search space can leave out the optimal solution, for large graphs this is not an issue. In most cases, a better solution is found in less time.

%----------------------------------------------------------------------------------------
%	REFERENCE LIST
%----------------------------------------------------------------------------------------

\bibliographystyle{plain}
\bibliography{bib}

%----------------------------------------------------------------------------------------

\end{document}
