%%%%%%%%%%%%%%%%%%%%%%%%%%%%%%%%%%%%%%%%%
% Journal Article
% LaTeX Template
% Version 1.4 (15/5/16)
%
% This template has been downloaded from:
% http://www.LaTeXTemplates.com
%
% Original author:
% Frits Wenneker (http://www.howtotex.com) with extensive modifications by
% Vel (vel@LaTeXTemplates.com)
%
% License:
% CC BY-NC-SA 3.0 (http://creativecommons.org/licenses/by-nc-sa/3.0/)
%
%%%%%%%%%%%%%%%%%%%%%%%%%%%%%%%%%%%%%%%%%

%----------------------------------------------------------------------------------------
%	PACKAGES AND OTHER DOCUMENT CONFIGURATIONS
%----------------------------------------------------------------------------------------

\documentclass[twoside,twocolumn]{article}

\usepackage{blindtext} % Package to generate dummy text throughout this template 

\usepackage[sc]{mathpazo} % Use the Palatino font
\usepackage[T1]{fontenc} % Use 8-bit encoding that has 256 glyphs
\linespread{1.05} % Line spacing - Palatino needs more space between lines
\usepackage{microtype} % Slightly tweak font spacing for aesthetics

\usepackage[english]{babel} % Language hyphenation and typographical rules

\usepackage[hmarginratio=1:1,top=32mm,columnsep=20pt]{geometry} % Document margins
\usepackage[hang, small,labelfont=bf,up,textfont=it,up]{caption} % Custom captions under/above floats in tables or figures
\usepackage{booktabs} % Horizontal rules in tables

\usepackage{lettrine} % The lettrine is the first enlarged letter at the beginning of the text

\usepackage{enumitem} % Customized lists
\setlist[itemize]{noitemsep} % Make itemize lists more compact

\usepackage{abstract} % Allows abstract customization
\renewcommand{\abstractnamefont}{\normalfont\bfseries} % Set the "Abstract" text to bold
\renewcommand{\abstracttextfont}{\normalfont\small\itshape} % Set the abstract itself to small italic text

\usepackage{titlesec} % Allows customization of titles
\renewcommand\thesection{\Roman{section}} % Roman numerals for the sections
\renewcommand\thesubsection{\roman{subsection}} % roman numerals for subsections
\titleformat{\section}[block]{\large\scshape\centering}{\thesection.}{1em}{} % Change the look of the section titles
\titleformat{\subsection}[block]{\large}{\thesubsection.}{1em}{} % Change the look of the section titles

\usepackage{fancyhdr} % Headers and footers
\pagestyle{fancy} % All pages have headers and footers
\fancyhead{} % Blank out the default header
\fancyfoot{} % Blank out the default footer
\fancyhead[C]{Running title $\bullet$ May 2016 $\bullet$ Vol. XXI, No. 1} % Custom header text
\fancyfoot[RO,LE]{\thepage} % Custom footer text

\usepackage{titling} % Customizing the title section

\usepackage{hyperref} % For hyperlinks in the PDF

%----------------------------------------------------------------------------------------
%	TITLE SECTION
%----------------------------------------------------------------------------------------

\setlength{\droptitle}{-4\baselineskip} % Move the title up

\pretitle{\begin{center}\Huge\bfseries} % Article title formatting
\posttitle{\end{center}} % Article title closing formatting
\title{Community Detection: Metaheuristic} % Article title
\author{%
\textsc{Jeroen Craps} \\[1ex] % Your name
\normalsize KU Leuven \\ % Your institution
\normalsize \href{mailto:jeroen.craps@student.kuleuven.be}{jeroen.craps@student.kuleuven.be} % Your email address
\and % Uncomment if 2 authors are required, duplicate these 4 lines if more
\textsc{Jorik De Waen} \\[1ex] % Second author's name
\normalsize KU Leuven \\ % Second author's institution
\normalsize \href{mailto:jorik.dewaen@student.kuleuven.be}{jorik.dewaen@student.kuleuven.be} % Second author's email address
}
\date{\today} % Leave empty to omit a date
%\renewcommand{\maketitlehookd}{%
%\begin{abstract}
%\noindent \blindtext % Dummy abstract text - replace \blindtext with your abstract text
%\end{abstract}
%}

%----------------------------------------------------------------------------------------

\begin{document}

% Print the title
\maketitle

%----------------------------------------------------------------------------------------
%	ARTICLE CONTENTS
%----------------------------------------------------------------------------------------

\section{Introduction}

\lettrine[nindent=0em,lines=3]{C} ommunities are found in networks everywhere in the world, due to the rise of social networks and the collection of other graph data over the last decade these networks have been growing exponentially.
Communities can be seen as fairly independent parts of the graph.
The information these graphs contain can be very valuable for a variety of reasons.
Faster algorithms are required to deal with this increase of data. 
The problem has been proven to be NP-hard \cite{Fortunato2010}.

%------------------------------------------------

\section{Overview}


Text requiring further explanation\cite{Pizzuti2009}.

%------------------------------------------------

\section{Methods}
Based on previous algorithms that have been used detect overlapping communities in large networks, the intention is to use techniques from outside of the Genetic Algorithm domain to improve the efficiency and performance of a recently used Multi-Agent approach \cite{multiagent2016}.
The main mtehod that will be used to facilitate is Node Clustering \cite{Ding2016}.
In the locus-based adjacency representation of the graph structure, all of the nodes are representated seperately which leads to a very large data structure when being used on large but realistic network graphs.
By clustering some of the obvious communities into new nodes prior to using the Genetic Algorithm, the space and time complexity should decrease.
Edge contraction based on node clustering should keep the general structure of the graph without much information loss if the relationship between the merged nodes in the graph is transitive.
Multiple methods of link clustering will be tested to see its performance compared to the original set-ups
A possible problem might be that by doing this is will become harder to find nodes that are in several communities.
Further research is required to see how this can be prevented while still improving the efficiency of the Genetic Algorithm.

%----------------------------------------------------------------------------------------
%	REFERENCE LIST
%----------------------------------------------------------------------------------------

\bibliographystyle{plain}
\bibliography{bib.bib}

%----------------------------------------------------------------------------------------

\end{document}
