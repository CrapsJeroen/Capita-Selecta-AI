\section{Introduction}\label{sec:introduction}

Networks are ever so present in the world, due to the rise of social media and the emergence of Big Data\footnote{Big data is a term for data sets that are so large or complex that traditional data processing applications are inadequate to deal with them.} over the last decade.
The detection of communities\footnote{A network is said to have community structure if the nodes of the network can be easily grouped into (potentially overlapping) sets of nodes such that each set of nodes is densely connected internally.} in these large networks has grown in importance.
Communities can be seen as fairly independent parts of the networks.
Overlap between communities can be compared to some people being part of multiple groups of friends on social networks at the same time.
A community implies important information about relationships between topology and network functionality.
The information contained in these networks can be of utmost importance to understanding the data that is being dealt with.
Problems related to community detection are often NP-hard.\\

Metaheuristics are a problem-independent algorithmic framework.
It provides a set of strategies to perform optimization by developing a certain heuristic.
A lot of work has been done with genetic algorithms \cite{Li2016, Newman2004}, which are one of the most common metaheuristics.
In this scientific report a method is presented to further help tackle this problem.
An attempt is made to decrease the amount of nodes and with that the length of chromosomes in the genetic algorithm.
With this we hope to improve the speed of the existing algorithms without decreasing their accuracy.
