\section{Introduction}\label{sec:introduction}

Networks are ever so present in the world, due to the rise of social media and the emergence of Big Data\footnote{Big data is a term for data sets that are so large or complex that traditional data processing applications are inadequate to deal with them.} over the last decade.
The detection of communities\footnote{A network is said to have community structure if the nodes of the network can be easily grouped into (potentially overlapping) sets of nodes such that each set of nodes is densely connected internally.} in these large networks has grown in importance.
Communities can be seen as fairly independent parts of the graph.
Certain elements can be present in multiple communities.
This concept can be compared to a person being part of multiple groups of friends on social networks at the same time.
A community implies important information about relationships between topology and network functionality.
The information contained in these graphs can be of utmost importance to understanding the data that is being dealt with.
The problem has been proven to be NP-hard \cite{Fortunato2010}.\\

Seeing that this problem is NP-hard the most interesting approach would be to use a meta-heuristic.
A lot of work has been done with genetic algorithms INSERT CITATIONS.
In this scientific report a method is presented to help tackle this problem.
An attempt is made to decrease the amount of nodes and with that the length of chromosones in the genetic algorithm.
With this we hope to improve the speed of the exsisting algorithms without decreasing their accuracy.
