\section{Background}
\label{chapter:background}
This chapter will explain some elements to be able to understand the contents of this scientific report.
The focus will be put on graph theory and genetic algorithms.
For both of these the basics will be shortly mentioned and the reasoning behind why they are important for this work.

\subsection{Graph Theory}
Graph theory is the mathematical theory of the properties and applications of graphs.
A graph $G=\left[V,E\right[$ is a collection of vertices (points) and edges (lines).
An edge connects two vertices or creates a loop for one vertice.
Graphs can have a certain properties, e.g. directed, undirected, weighted or complete.
Certain combinations of these properties are also possible.

In directed graphs an edge has an orientation, e.g. $A \rightarrow B$. This says that A has a connection to B, but B does not have a connection to A.
Normally in an undirected graph an edge might look like $A - B$, which implies a connection in both ways.

Edges in the graph can contain a certain scoring-function, if this is the case then it's called a weighted graph.
An common example of this is the euclidean distance between the two vertices the edge connects together.
A graph is complete when all nodes are connected to eachother.
A graph $G'=\left[V',E'\right[$ is a subgraph of $G$ if $V' \subset V$, $E' \subset E$ and $E'$ only including edges between the vertices of $V'$.
A complete subgraph is called a \textbf{clique}.
There are other properties for graphs, but they are not required for the understanding of this report.

\subsection{Genetic Algorithm}
