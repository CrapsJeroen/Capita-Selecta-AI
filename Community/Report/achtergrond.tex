\section{Background}
\label{sec:background}
This chapter will explain certain elements to fully be able to understand the contents of this scientific report.
The focus will be put on graph theory and genetic algorithms.

\subsection{Graph Theory}
Graph theory is the mathematical theory of the properties and applications of graphs \cite{Demoen}.
A graph $G=\lbrace V,E \rbrace$ is a collection of vertexes (points) and edges (lines).
An edge connects two vertexes or creates a loop for a single vertex to itself.
Graphs can have a certain properties, e.g. directed, undirected, weighted or complete.
Certain combinations of these properties are also possible.

In directed graphs an edge has an orientation, e.g. $A \rightarrow B$. This says that A has a connection to B, but B does not have a connection to A.
Normally in an undirected graph an edge might look like $A - B$, which implies a connection in both directions. Edges in the graph can contain a certain weight or score, if this is the case then it's called a weighted graph.
An common example of this is the euclidean distance between the two vertexes which the edge connects together.

A graph is called complete when all nodes are connected to eachother \cite{Demoen}.
A graph $G'=\lbrace V',E' \rbrace$ is a subgraph of $G$ if $V' \subset V$, $E' \subset E$ and $E'$ only including edges between the vertexes of $V'$.
A complete subgraph is called a \textbf{clique}.
There are other properties for graphs, but these are not required for the understanding of this report.

\subsection{Genetic Algorithm}
In this subsection the concept of genetic algorithms will be expanded upon \cite{genetics}. 
First, the basic concept will be explained. 
Afterwards a more concrete explanation will be given about the elements that have been used in this research.
Important to note is that if a problem can be represented and the proper operators can be designed any problem can be solved by a genetic algorithm if the purpose of it is to optimize a certain scoring-function.

\subsubsection{Basics}
A genetic algorithm is a metaheuristic inspired by natural evolution and selection.
The goal of the metaheuristic is to mimic biological evolution, which strives to improve upon itself \cite{genetics}.
It is a method for solving optimization problems which can be constrained or unconstrained.
A new population of individuals is generated each iteration.
Which individuals survive to the next generation depends on the survivor policy that is selected by the user.
The strongest individual, according to the fitness function, approaches an optimal solution for the given problem.

A typical genetic algorithm contains a genetic representation of the solution domain and a fitness function to evaluate the solution in the problem domain.
The chromosome\footnote{Also known as the genotype.} is the set parameters which define a solution for the problem.
A set of chromosomes is called a population.
By updating the population on a iterative basis by using certain operators, the algorithm attempts to find the optimal solution for the specified problem.

\subsubsection{Representation}
The most commonly used representation in genetic algorithms is the binary representation.
But this isn't always a good choice, because the translation to a representation that is easily understandable might be very difficult.
That is why for the problem that will be discussed in this report we will be using a locus-based representation \cite{Li2016}.
An example of this can be seen in Fig. \ref{figure:locus}.

\begin{figure}
\begin{center}
\includegraphics[width=0.6\textwidth]{images/locus.png}
\caption{The locus-based representation of a graph.}\label{figure:locus}
\end{center}
\end{figure}

An individual is represented as an list of numbers. 
The index represents the identification number of a vertex in the graph.
The value in the array stands for the vertex it is connected to.
While decoding the representation we say that a set of vertexes are in a community if there is a link between any of the vertexes.
A restriction of this representation is that vertexes can only be part of a single community.
While it is common for communities to have overlap, it remains quite hard to deal with in a compact representation.

\begin{figure}
\begin{center}
\includegraphics[width=0.6\textwidth]{images/encode.png}
\caption{The visualisation of Fig. \ref{figure:locus}.}\label{figure:encode}
\end{center}
\end{figure}

\subsubsection{Operators}
The goal of the algorithm is to improve on the initial population which can be initiated randomly or with a certain heuristic.
To do so some operators are required \cite{genetics}.
Exploring the search space in usually done by using mutation which will randomly change an individual.
Exploitation is done by combining information of two (or more) individuals to one (or more) new individual(s).
\newpage
\subsubsection*{Crossover}
Crossover is the operator that will attempt to find the optimum in the current population by combining high scoring individuals with eachother \cite{genetics}.
The exact parent selection is again something that has to be defined by the user, otherwise a random selection will occur.
It tries to combine the superior parameters from both parents and uses these to form an even better individual.
Combining two good solutions doesn't always lead to a better one, but combining one good solution with a bad one might do so.
Therefor it is important to take lesser solutions still into account.

Some example of crossovers on binary representations are the following:
\begin{itemize}
\item In one-point crossover a random crossover point is selected. The individuals are split into two parts: left and right. All of the information on the left side of the first individual will be combined with the information of the right side of the second individual. A simple example of this can be seen below.
\begin{center}
\begin{tabular}{|c|c|c|c|c|c|}
\hline
0 & 1 & 0 & \textbf{0} & \textbf{0} & \textbf{1} \\ \hline \hline
0 & 1 & 0 & \textbf{1} & \textbf{1} & \textbf{0} \\ \hline
\end{tabular}
\\
$\:$ \\
$\downarrow$ \\
$\:$ \\
\begin{tabular}{|c|c|c||c|c|c|}
\hline
0 & 1 & 0 & \textbf{1} & \textbf{1} & \textbf{0} \\ \hline \hline
0 & 1 & 0 & \textbf{0} & \textbf{0} & \textbf{1} \\ \hline
\end{tabular}
\end{center}
\item In two-point crossover two random crossover points are selected. All of the information between these two points will be exchanged between the two individuals. Again a simple example is shown here.
\begin{center}
\begin{tabular}{|c|c|c|c|c|c|}
\hline
1 & 0 & \textbf{1} & \textbf{1} & 0 & 0 \\ \hline \hline
0 & 1 & \textbf{0} & \textbf{1} & 1 & 0 \\ \hline
\end{tabular}
\\
$\:$ \\
$\downarrow \qquad \quad \downarrow$ \\
$\:$ \\
\begin{tabular}{|c|c||c|c||c|c|}
\hline
1 & 0 & \textbf{0} & \textbf{1} & 0 & 0 \\ \hline \hline
0 & 1 & \textbf{1} & \textbf{1} & 1 & 0 \\ \hline
\end{tabular}
\end{center}
\end{itemize}

The probability of crossover is in practice chosen quite high, because this is the main exploiting operator.
Without crossover it seems to be very difficult to keep improving on the current population.
There are several mechanisms in place to adjust this probability during the process, but are not mandatory to use.

\subsubsection*{Mutation}
Mutation normally allows for new sections of the search space to be explored by creating a new value for a parameter that previously wasn't present in the population \cite{genetics}.
An example of this is a bit flip on a binary representation for a possible solution.
By doing this a bit that is always $0$ in the population can be changed to $1$ and thus introducing new ``DNA'' into the population.
This is necessary to keep the population from going into a local optima and to help explore the entire search space without potentially missing out on a better solution. \\

\begin{center}
\begin{tabular}{|c|c|c|c|c|c|}
\hline
1 & 0 & \textbf{1} & 1 & 0 & 0 \\ 
\hline
\end{tabular}
\\
$\:$ \\
$\quad \downarrow \: \qquad \:$ \\
$\:$ \\
\begin{tabular}{|c|c|c|c|c|c|}
\hline
1 & 0 & \textbf{0} & 1 & 0 & 0 \\ \hline
\end{tabular}
\end{center}

In practice the amount of mutations is randomly chosen as well as the bits that are being flipped.
The example shows only a single bit flip.
Probability of a mutation happening is normally chosen quite low as it can disrupt the current exploitation process.
This value can be increased when a population is growing quite stale.