\section{Conclusion}
\label{sec:conclusion}
Our experiments show that the effectiveness of reducing the graph is highly dependent on how interconnected the graph is. The advantage of reducing the graph is not only in execution time, as we expected, but it also significantly improves the quality of the initial random population for the genetic algorithm. This advantage is clear in graphs with on the order of hundreds of vertexes or more, and proportional with the amount reduction that was done. There are also obvious performance benefits for medium-sized graphs of several thousands of vertexes. However, for larger graphs with tens of thousands of vertexes, the genetic algorithm becomes ineffective when combined with the short execution times we allowed. The reduced graph still has a significant advantage, but only due to better initialization.
\par
Because of this we conclude that our graph reduction preprocessing can be very useful in certain scenarios. When graphs are highly interconnected, reducing the graph improves both the execution time and the score of the result. While the reduced search space can leave out the optimal solution, for large graphs this is not an issue. In most cases, a better solution is found in less time.